\documentclass{article}
\usepackage{silence}
\WarningsOff[everypage]%
\usepackage{scimisc-cv}


\title{Sahil Modi Resume}
\author{Sahil Modi}
\date{Aug 2020}

%% These are custom commands defined in scimisc-cv.sty
\cvname{Sahil Modi}
\cvpersonalinfo{
    \email{smodi9@illinois.edu}
    \phone{847-890-3506}
    \location{U.S. Citizen}
    \homepage{sahilmodi.dev}
    \linkedin{sahil-modi}
    \github{sahilmodi}
}

\begin{document}

%% QR Code
\backgroundsetup{opacity=1, scale=1, angle=0, contents={%
\begin{tikzpicture}[remember picture,overlay,shift={(current page.north east)}]
\node[anchor=north east,xshift=-1.1cm,yshift=-0.4cm]{\includegraphics[width=1.5cm]{qr_code.png}};
\end{tikzpicture}}}
\BgThispage

% \maketitle %% This is LaTeX's default title constructed from \title,\author,\date

\makecvtitle %% This is a custom command constructing the CV title from \cvname, \cvpersonalinfo

\section{Education}
\cvsubsection{M.S. Computer Science}[May 2022]
[University of Illinois at Urbana-Champaign, Thesis Topic: Computer Vision \& Deep Learning][GPA: 4.00/4.00]
\vspace{0.2cm}
\cvsubsection{B.S. Computer Science, Minor in Statistics}[May 2021]
[University of Illinois at Urbana-Champaign][GPA: 3.96/4.00]
\begin{description}[widest=Coursework]
    \item[Coursework]	Computational Photography, Computer Vision, Machine \& Deep Learning
    \item[Awards]	\$5000 TechnipFMC \& \$3000 TBP Hayward Scholarships,  2\textsuperscript{nd} place at PygHacks, 2\textsuperscript{nd} place at Clorox competition
\end{description}

\section{Technical Skills}
\begin{description}[widest=Frameworks]
    \item[Languages]	Python, C++, C, Java, Javascript, Typescript, SQL, Bash
    \item[Frameworks]	PyTorch, OpenCV, TensorFlow, Linux, Git
\end{description}

\section{Professional Experience}

%% Another custom command provide by scimisc-cv.sty.
%% First two arguments are typeset on the first line in bold; 3rd and 4th arguments are typeset on second line in italics. 2nd, 3rd and 4th arguments are OPTIONAL

\cvsubsectionsingle{NVIDIA}[May 2021 -- Aug 2021]
[Software Engineering Intern, TensorRT][Santa Clara, CA]

\begin{itemize}
    % \item Collaborated in a 6 person team to revise frontend asset generation
    \item Introduced a software-based kernel timing heuristic (DLSim) for \textbf{neural network optimization}
    \item Bridged \textbf{C++} TensorRT and \textbf{Python} DLSim by implementing a server-client interface handling 10s of queries/s with convolutional layer parameter translation
    \item Compared 5 networks (\textbf{ResNet-50}, \textbf{MobileNet}, \textbf{Inception-V4}, etc), across 3 batch sizes and FP16, INT8 precision, meeting the <10\% throughout reduction goal
\end{itemize}

\cvsubsectionsingle{Amazon}[May 2020 -- Aug 2020]
[Software Development Engineer Intern][Seattle, WA]

\begin{itemize}
    % \item Collaborated in a 6 person team to revise frontend asset generation
    \item Reduced aggregate \textbf{Javascript} asset build time by 18.5\% and decreased memory usage by 11\%  
    \item Analyzed code syntax trees for unfavorable behavior, decreasing final asset size by 5\%
    \item Designed a variant generation algorithm an \textbf{order of magnitude faster} for server built variants and client responsive variants 
\end{itemize}

%% An example of leaving an argument empty
\cvsubsection{Distributed Autonomous Systems Laboratory}[Jan 2020 -- May 2020]
[Undergraduate Research Assistant | Advisors: Dr. Girish Chowdhary, Dr. Saurabh Gupta][Urbana, IL]
\begin{itemize}
    \item Investigated vision-based robot heading estimation with a \textbf{self-supervised network} on \textbf{PyTorch} achieving 2 degrees error 
    \item Devised a \textbf{supervised network} for autonomously calculate pose and drive a robot with distance to intervention of 30 meters
    \item Augmented video data with \textbf{homographic transformations} to simulate robot variance and increase dataset coverage
\end{itemize}

\cvsubsectionsingle{EarthSense}[Sep 2019 -- Dec 2019]
[Computer Vision Research Intern | Advisor: Dr. Girsh Chowdhary][Champaign, IL]
\begin{itemize}
    \item Ascertained intrinsic camera matrices of Terrasentia robot cameras
    \item Achieved 92\% accuracy for corn ear height estimation from video by fusing a \textbf{neural network} with \textbf{single view metrology}
\end{itemize}

\cvsubsectionsingle{Northrop Grumman}[May 2019 -- Aug 2019]
[Software Engineering Intern][Rolling Meadows, IL]
\begin{itemize}
    \item Developed a C\# application to configure and test missile warning algorithms
    and pulled in project schedule by 2 months 
\end{itemize}

\cvsubsectionsingle{EarthSense}[Sep 2018 -- May 2019]
[Computer Vision Intern][Champaign, IL]
\begin{itemize}
    \item Trained a \textbf{convolutional neural network} with \textbf{TensorFlow} on a biased dataset to classify lodging of wheat with 80\% accuracy
    \item Deployed a \textbf{TensorFlow ML} model to detect and count plant stems with 96\% accuracy
% \item Proposed additional data collection metrics to address overfitting
\end{itemize}

\cvsubsectionsingle{Swarm Robotix}[May 2018 -- Aug 2018]
[Software Engineering Intern][Naperville, IL]
\begin{itemize}
    \item Collaborated with 2 people to create vision algorithms with \textbf{OpenCV} to detect shipping container corner castings
\end{itemize}

\section{Project Highlights}
\cvproject{HackIllinois Stock Analysis}
\begin{itemize}
    \item A python package discovering sentiment about a company from its tweets using NLTK and correlating it with stock price
    \item \textbf{Linear, ridge regression}, and a \textbf{convolutional neural network} are used for prediction and compared against each other
\end{itemize}
 \cvproject{CU-Recycle}
\begin{itemize}
    \item Devised an Android application to report an item's recyclability status in the Urbana-Champaign area, \textbf{winning 2\textsuperscript{nd}} at PygHacks 
    \item Trained a convolutional neural network for \textbf{object recognition} with \textbf{Keras} to overcome lighting and object variance
\end{itemize}

\section{Leadership}
\cvleadership{CS 461: Computer Security}[Course Assistant][Jan 2021 -- May 2021]
% \begin{itemize}
%     \item Held office hours to answer student questions about course material
% \end{itemize}
\cvleadership{CS 374: Algorithms}[Course Assistant][Aug 2020 -- Dec 2020]
% \begin{itemize}
%     \item Assisted graduate TAs with office hours and graded student homework
% \end{itemize}
\cvleadership{DGS Student Council}[Class Representative][Aug 2019 -- May 2021]
% \begin{itemize}
%     \item Providing advice to Division of General Studies director on pre-engineering events and information
%     % \item Serving on engineering student panel for prospective and incoming students 
% \end{itemize}
\cvleadership{Engineering Freshmen Council}[IT Chair][Aug 2017 -- May 2018]
% \begin{itemize}
%     \item Redesigned Engineering Freshmen Council website
% \end{itemize}
% \cvleadership{iRobotics MRDC}[Software Lead][2017 -- 2018]
% \begin{itemize}
%     \item Managed a team of 3 to design and develop the robot's software stack
% \end{itemize}

\section{Publications}
\cvleadership{Learned Visual Navigation for Under-Canopy Agricultural Robots}[Robotics: Science and Systems][2021]
Arun Sivakumar, \textbf{Sahil Modi}, Mateus Gasparino, Che Ellis, Andres Velasquez, Girish Chowdhary, Saurabh Gupta

\end{document}
